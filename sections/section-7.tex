\section{Conclusions 结论}

\begin{paracol}{2}

    In this paper, a global clay database is presented, and 24 transformation models among clay parameters in the literature are investigated. This database contains clays with a wide range of $S_t$, OCR, and PI values and from a wide range of geographical locales. It is found that most of the 24 models fit to the data trends of the global database. There are few exceptions, and it is believed that the poor fit is due to the fact that these models were developed by databases with a limited range of clay types (e.g., no quick clays are in the database or only clays of a single region were in the database).
        
    \switchcolumn
    
    本文提出了一个全球黏土数据库,并研究了文献中24个黏土参数之间的转换模型。 该数据库包含具有广泛的$S_t$,OCR和PI值且来自不同地理位置的粘土。 发现这24个模型中的大多数都适合于全球数据库的数据趋势。 几乎没有例外,并且认为拟合度较差是由于以下事实:这些模型是由黏土类型范围有限的数据库开发的(例如,数据库中没有快速黏土,或者只有单个区域的黏土在数据库中)。
        
    \switchcolumn*
    
    The global database is further used to calibrate the biases and uncertainties for the models in literature. It is found that more recent models tend to have smaller biases. Also, the uncertainties calibrated by the global database are mostly larger than the uncertainties reported in the literature. The large uncertainties may be reduced by considering PI and $S_t$ as secondary input parameters for the models. The biases are also updated after incorporating PI and $S_t$.
        
    \switchcolumn
    
    全球数据库还用于校准文献中模型的偏差和不确定性。 已经发现,较新的模型倾向于具有较小的偏差。 此外,由全球数据库校准的不确定性大多大于文献中报道的不确定性。 通过将PI和$S_t$作为模型的辅助输入参数,可以减少较大的不确定性。 在合并PI和$S_t$之后,偏差也会更新。
        
\end{paracol}