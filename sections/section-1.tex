\section{Introduntion 简介}

<<<<<<< HEAD
\begin{paracol}{2}

    Geotechnical variability is a complex attribute that needs careful evaluation. \citet{Phoon1999612} demonstrated using fairly extensive soil statistics that geotechnical variability depends on the site condition, measurement error associated with a field test, and quality of the correlation model adopted to relate the field test to a design property. The first component refers to inherent soil variability, which is customarily categorized as aleatoric in nature because it cannot be reduced by performing more tests. The second and third components, namely measurement error inevitably introduced in a test procedure and data scatter about a mean correlation trend (typically in the form of a linear regression equation), are customarily categorized epistemic in nature. They can be reduced by gathering more data or building better models. While there are merits to categorizing uncertainties as aleatoric or epistemic, one should be mindful that this demarcation is in part a modeler’s choice \citep{DerKiureghian2007}. From a practical perspective, it is perhaps more important to align tatistical characterization to the property evaluation procedure already embedded in our geotechnical engineering practice. In recognition of the need to respect sound geotechnical engineering practice, \citet{Phoon1999625} presented guidelines for coefficients of variation (COVs) of some soil properties as a function of the test method, correlation equation, and soil type. The key conclusion in this study is that it is not possible to assign a single coefficient of variation (COV) to a design property, such as the undrained shear strength. Geotechnical reliability-based design (RBD) equations that are calibrated using this single COV assumption are too simplistic, because diverse methodologies in estimating soil properties are ignored. This diversity is actually good practice, because there is a need to accommodate diverse site conditions. \citet{Phoon1999612,Phoon1999625} advocated that the calibration of geotechnical RBD equations should be carried out in explicit recognition of property variability and in full compliance with how soil properties are physically evaluated in practice. The framework recommended by \citet{Phoon1999612,Phoon1999625} and the ensuing three-tier classification scheme of soil property variability \citep{Phoon2008344} should be viewed as the minimum requirements in variability characterization. The third edition of ISO 2394 (General principles on reliability for structures; to be published in 2015) includes a new Annex D on “Reliability of geotechnical structures” where the importance of respecting sound geotechnical engineering practices in variability characterization and reliability calibration is strongly emphasized.

    \switchcolumn

    岩土变异性是一个复杂的属性,需要仔细评估。 \citet{Phoon1999612}使用相当广泛的土体统计数据证明,岩土工程变异性取决于现场条件,与现场测试相关的测量误差以及将现场测试与设计属性相关联的相关模型的质量。第一部分是土体固有的变异性,由于无法通过执行更多的测试来减少,因此习惯上被归类为无酸。第二和第三部分,即不可避免地在测试过程中引入的测量误差和关于平均相关趋势的数据散布(通常以线性回归方程的形式),本质上通常被归类为认识论。可以通过收集更多数据或建立更好的模型来减少它们。尽管将不确定性归类为偶然性或认知性是有好处的,但应该记住,这种划分在某种程度上是建模者的选择\citep{DerKiureghian2007}。从实践的角度来看,将地物特征与已经嵌入到我们的岩土工程实践中的属性评估程序对齐可能更为重要。考虑到需要尊重合理的岩土工程实践,\citet{Phoon1999625}提出了一些土体特性的变异系数(COV)的准则,这些准则是测试方法,相关方程和土体类型的函数。这项研究的关键结论是,不可能为设计特性(例如不排水的剪切强度)分配单个变异系数(COV)。使用这种单一COV假设进行校准的基于岩土工程的基于可靠性的设计(RBD)方程过于简单,因为忽略了估算土体性质的多种方法。实际上,这种多样性是一种好习惯,因为有必要适应各种现场条件。 \citet{Phoon1999612,Phoon1999625}提倡,对岩土工程RBD方程的标定应在明确认识到物性变异性的前提下进行,并完全符合实际中对土体物性的评估方法。 \citet{Phoon1999612,Phoon1999625}建议的框架和随后的三层土体性质变异性分类方案\citep{Phoon2008344}应被视为变异性表征的最低要求。 ISO 2394第三版(关于结构可靠性的一般原则;将于2015年发布)包括有关“岩土结构可靠性”的新附件D,在该附录D中,强烈强调了在变异性表征和可靠性校准中尊重岩土工程实践的重要性。

    \switchcolumn*

    The characterization of geotechnical variability is far from being a mature area in research. An astute practitioner would readily point out that multiple tests are commonly conducted in a site investigation and it is common practice to estimate a design property from these tests, either by straightforward averaging or picking a credible worst-case value from the range of values produced by different tests. The information collected in a site investigation programme is fundamentally multivariate in nature and this aspect has not been considered in the earlier studies mentioned above. The purpose of this paper is to develop unbiased transformation models and to quantify their associated uncertainties for 10 common clay parameters. A companion paper \citep{Ching2014686} develops a multivariate probability model coupling these clay parameters. The supporting database contains information from multiple tests that are collected in close proximity. Inother words, each data point records soil information at a specific location and depth, i.e., at a specific sampling point. Note that measurement error is present, but it is not possible to isolate measurement error from transformation uncertainty in conventional site investigation programs. Hence, the transformation uncertainties presented in this study include some measurement errors, but these errors are relatively minor for cone penetration testing with pore pressure measurement (CPTU) \citep{Phoon1999612}. Inherent soil variability is clearly not considered in this study. In principle, inherent soil variability can be incorporated by extending the multivariate probability model (whichapplies to a sampling “point”) to a vector random field covering the three-dimensional (3D) spatial domain of the entire site. The outcomes of this study are thus incomplete in this sense, but they can be viewed as paving the way for characterization of geotechnical variability to advance beyond univariate data and to achieve closer alignment to how soil properties are estimated in actual practice from site investigation programs.

    \switchcolumn

    岩土工程变异性的表征还远远不是研究的成熟领域。精明的从业人员会很容易地指出,在现场调查中通常会进行多次测试,并且通常的做法是根据这些测试来评估设计属性,方法是直接取平均值,或者从测试产生的值的范围内选择可信的最坏情况下的值。不同的测试。在现场调查程序中收集的信息本质上本质上是多元的,并且在上述早期研究中并未考虑到这方面。本文的目的是为10个常见黏土参数建立无偏转换模型并量化其相关的不确定性。伴随论文\citep{Ching2014686}开发了耦合这些粘土参数的多元概率模型。支持数据库包含来自紧密收集的多个测试的信息。换句话说,每个数据点记录特定位置和深度,即特定采样点的土体信息。请注意,存在测量误差,但是在传统的站点调查程序中无法将测量误差与变换不确定性区分开。因此,本研究中提出的变换不确定性包括一些测量误差,但是对于使用孔隙压力测量(CPTU)进行的圆锥渗透测试,这些误差相对较小\citep{Phoon1999612}。这项研究显然没有考虑土体固有的变异性。原则上,可以通过将多元概率模型(适用于采样“点”)扩展到覆盖整个站点的三维(3D)空间域的向量随机字段来合并固有的土体可变性。因此,从这个意义上说,这项研究的结果是不完整的,但可以认为它们为表征岩土工程变异性铺平了道路,使岩土变异性超越了单变量数据,并与现场调查程序在实际实践中对土体特性的估算更加接近。

    \switchcolumn*

    As noted by \citet{Phoon1999612}, the measurement from a geotechnical test is typically not directly applicable to design. Instead, a transformation model is needed to relate the test measurement to an appropriate design parameter. Most transformation models in geotechnical engineering are obtained by empirical or semi-empirical data-fitting using regression analyses. These transformation models are widely adopted in geotechnical engineering practice as a matter of practical expediency. Useful compilations of these models (mostly pairwise correlations) are available in the literature (e.g., \citealp{Kulhawy1990}; \citealp{Mayne2001}). A cursory review of these compilations would reveal a rather bewildering variety and number of models. Most models were developed for a specific geomaterial type and (or) a specific locale.

    \switchcolumn

    正如\citet{Phoon1999612}指出的那样,岩土试验的测量结果通常不适用于设计。 取而代之的是,需要一个转换模型来将测试度量与适当的设计参数相关联。 岩土工程中的大多数转换模型都是通过使用回归分析的经验或半经验数据拟合获得的。 出于实际考虑,这些转换模型已在岩土工程实践中广泛采用。 这些模型的有用汇编(主要是成对相关)可在文献中找到(例如\citealp{Kulhawy1990}; \citealp{Mayne2001})。 粗略地回顾一下这些汇编,将会发现令人困惑的各种模型。 大多数模型是针对特定的土工材料类型和(或)特定的区域设置而开发的。

    \switchcolumn*

    It is not judicious to apply these models indiscriminately to other sites without a proper appreciation of geomaterial behaviour and geology. This “site-specific” limitation is a distinctive and fundamental feature of geotechnical engineering practice. Geotechnical design must take cognizance of this limitation to avoid gross oversimplification of “ground truths.” As opposed to site-specific models, \citet{Ching201252} demonstrated the construction of “global” models. Global models are calibrated from global databases covering many sites and geomaterial types. \citet{Ching201252} observed that site-specific models are generally more precise than global models, but they can be significantly biased when applied to another site. On the contrary, global models are less precise than site-specific models, but they are less biased. Their observations are already well appreciated by engineers. The key contribution from \citet{Ching201252} was to demonstrate these observations with statistical rigor using a sizeable global database.

    \switchcolumn

    在没有适当了解地球物质行为和地质的情况下,不加选择地将这些模型应用于其他地点并不明智。这种“特定于地点”的限制是岩土工程实践的独特和基本特征。岩土设计必须意识到这一局限性,以避免对“地面真相”的过分简化。与特定地点的模型相反,\citet{Ching201252}演示了“全局”模型的构建。全局模型是根据覆盖许多站点和岩土材料类型的全局数据库进行校准的。 \citet{Ching201252}观察到,特定于站点的模型通常比全局模型更为精确,但将其应用于其他站点时可能会出现明显偏差。相反,全局模型不如特定于站点的模型精确,但它们的偏向性较小。他们的观察已经得到工程师的赞赏。\citet{Ching201252}的主要贡献是使用庞大的全球数据库以严格的统计数据证明了这些观察结果。

    \switchcolumn*

    Because most transformation models were built based on their own databases, their ranges of application are, in principle, limited to the range of characteristics contained in the databases, e.g., certain soil types, certain range of soil properties (e.g., insensitive clays), and certain geographic locations. It is important to assess their biases and the uncertainties when these models are applied globally (i.e., applying these models outside their range of calibration). In the current paper, a global clay database is compiled and presented. This database consists of data points from 251 studies, covering clay data from 30 regions or countries worldwide. Hence, the range of data covered by this “global” database is broader than that underlying the calibration of existing transformation models in the literature. Ten parameters of clays are of main interest, including three index properties (i.e., Atterberg’s limits); four parameters for effective stresses, shear strength, and sensitivity; and three parameters from piezocone tests (CPTU). This global database is the largest database compiled by the authors thus far in terms of number of data points and number of parameters of interest. \autoref{table:1} shows the databases compiled by the authors, labeled as (soil type)/(number of parameters of interest)/(number of data points). The current global database is CLAY/10/7490. The first purpose of this paper is to present this large database and verify whether the correlation behaviours in the data points are consistent with those exhibited by existing transformation models in the literature. Most of these models are sitespecific models. The biases and uncertainties in these models will be estimated using the global database. The site-specific models can be applied to a wider range of conditions when their biases are corrected and their transformation uncertainties are  uitably revised.

    \switchcolumn

    由于大多数转化模型都是基于其自身的数据库构建的,因此,其应用范围原则上仅限于数据库中包含的特征范围,例如某些土体类型,某些土体特性范围(例如不敏感的粘土),和某些地理位置。在全球范围内应用这些模型时(即在校准范围之外应用这些模型),评估它们的偏差和不确定性很重要。在当前的论文中,一个全球粘土数据库被编译并呈现出来。该数据库包含来自251个研究的数据点,涵盖了来自全球30个地区或国家的粘土数据。因此,该“全球”数据库所涵盖的数据范围比文献中现有转换模型的标定所依据的范围要广。黏土的10个主要参数是我们关注的重点,其中包括三个指数属性(即,阿特伯格极限);有效应力,剪切强度和灵敏度的四个参数;以及压电锥测试(CPTU)的三个参数。就数据点的数量和感兴趣的参数的数量而言,该全局数据库是迄今为止作者编写的最大的数据库。\cntableref{table:1}显示了作者编译的数据库,标记为(土体类型)/(感兴趣的参数数量)/(数据点数量)。当前的全局数据库是CLAY/10/7490。本文的首要目的是介绍这个大型数据库,并验证数据点中的相关行为是否与文献中现有转换模型所显示的行为一致。这些模型大多数是特定于站点的模型。这些模型中的偏差和不确定性将使用全球数据库进行估算。当特定位置的模型的偏差得到纠正并且适当地修改了转换不确定性时,可以将其应用于更广泛的条件。

=======
\begin{Parallel}{0.60\textwidth}{}
    \ParallelLText
    {
        Geotechnical variability is a complex attribute that needs careful evaluation. \citet{Phoon1999612} demonstrated using fairly extensive soil statistics that geotechnical variability depends on the site condition, measurement error associated with a field test, and quality of the correlation model adopted to relate the field test to a design property. The first component refers to inherent soil variability, which is customarily categorized as aleatoric in nature because it cannot be reduced by performing more tests. The second and third components, namely measurement error inevitably introduced in a test procedure and data scatter about a mean correlation trend (typically in the form of a linear regression equation), are customarily categorized epistemic in nature. They can be reduced by gathering more data or building better models. While there are merits to categorizing uncertainties as aleatoric or epistemic, one should be mindful that this demarcation is in part a modeler’s choice \citep{DerKiureghian2007}. From a practical perspective, it is perhaps more important to align tatistical characterization to the property evaluation procedure already embedded in our geotechnical engineering practice. In recognition of the need to respect sound geotechnical engineering practice, \citet{Phoon1999625} presented guidelines for coefficients of variation (COVs) of some soil properties as a function of the test method, correlation equation, and soil type. The key conclusion in this study is that it is not possible to assign a single coefficient of variation (COV) to a design property, such as the undrained shear strength. Geotechnical reliability-based design (RBD) equations that are calibrated using this single COV assumption are too simplistic, because diverse methodologies in estimating soil properties are ignored. This diversity is actually good practice, because there is a need to accommodate diverse site conditions. \citet{Phoon1999612,Phoon1999625} advocated that the calibration of geotechnical RBD equations should be carried out in explicit recognition of property variability and in full compliance with how soil properties are physically evaluated in practice. The framework recommended by \citet{Phoon1999612,Phoon1999625} and the ensuing three-tier classification scheme of soil property variability \citep{Phoon2008344} should be viewed as the minimum requirements in variability characterization. The third edition of ISO 2394 (General principles on reliability for structures; to be published in 2015) includes a new Annex D on “Reliability of geotechnical structures” where the importance of respecting sound geotechnical engineering practices in variability characterization and reliability calibration is strongly emphasized.
    }
    \ParallelRText
    {
        岩土变异性是一个复杂的属性,需要仔细评估。 \citet{Phoon1999612}使用相当广泛的土体统计数据证明,岩土工程变异性取决于现场条件,与现场测试相关的测量误差以及将现场测试与设计属性相关联的相关模型的质量。第一部分是土体固有的变异性,由于无法通过执行更多的测试来减少,因此习惯上被归类为无酸。第二和第三部分,即不可避免地在测试过程中引入的测量误差和关于平均相关趋势的数据散布(通常以线性回归方程的形式),本质上通常被归类为认识论。可以通过收集更多数据或建立更好的模型来减少它们。尽管将不确定性归类为偶然性或认知性是有好处的,但应该记住,这种划分在某种程度上是建模者的选择\citep{DerKiureghian2007}。从实践的角度来看,将地物特征与已经嵌入到我们的岩土工程实践中的属性评估程序对齐可能更为重要。考虑到需要尊重合理的岩土工程实践,\citet{Phoon1999625}提出了一些土体特性的变异系数(COV)的准则,这些准则是测试方法,相关方程和土体类型的函数。这项研究的关键结论是,不可能为设计特性(例如不排水的剪切强度)分配单个变异系数(COV)。使用这种单一COV假设进行校准的基于岩土工程的基于可靠性的设计(RBD)方程过于简单,因为忽略了估算土体性质的多种方法。实际上,这种多样性是一种好习惯,因为有必要适应各种现场条件。 \citet{Phoon1999612,Phoon1999625}提倡,对岩土工程RBD方程的标定应在明确认识到物性变异性的前提下进行,并完全符合实际中对土体物性的评估方法。 \citet{Phoon1999612,Phoon1999625}建议的框架和随后的三层土体性质变异性分类方案\citep{Phoon2008344}应被视为变异性表征的最低要求。 ISO 2394第三版(关于结构可靠性的一般原则;将于2015年发布)包括有关“岩土结构可靠性”的新附件D,在该附录D中,强烈强调了在变异性表征和可靠性校准中尊重岩土工程实践的重要性。
    }
    \ParallelPar
    \ParallelLText
    {
        The characterization of geotechnical variability is far from being a mature area in research. An astute practitioner would readily point out that multiple tests are commonly conducted in a site investigation and it is common practice to estimate a design property from these tests, either by straightforward averaging or picking a credible worst-case value from the range of values produced by different tests. The information collected in a site investigation programme is fundamentally multivariate in nature and this aspect has not been considered in the earlier studies mentioned above. The purpose of this paper is to develop unbiased transformation models and to quantify their associated uncertainties for 10 common clay parameters. A companion paper \citep{Ching2014686} develops a multivariate probability model coupling these clay parameters. The supporting database contains information from multiple tests that are collected in close proximity. Inother words, each data point records soil information at a specific location and depth, i.e., at a specific sampling point. Note that measurement error is present, but it is not possible to isolate measurement error from transformation uncertainty in conventional site investigation programs. Hence, the transformation uncertainties presented in this study include some measurement errors, but these errors are relatively minor for cone penetration testing with pore pressure measurement (CPTU) \citep{Phoon1999612}. Inherent soil variability is clearly not considered in this study. In principle, inherent soil variability can be incorporated by extending the multivariate probability model (whichapplies to a sampling “point”) to a vector random field covering the three-dimensional (3D) spatial domain of the entire site. The outcomes of this study are thus incomplete in this sense, but they can be viewed as paving the way for characterization of geotechnical variability to advance beyond univariate data and to achieve closer alignment to how soil properties are estimated in actual practice from site investigation programs
    }
    \ParallelRText
    {
        岩土工程变异性的表征还远远不是研究的成熟领域。精明的从业人员会很容易地指出,在现场调查中通常会进行多次测试,并且通常的做法是根据这些测试来评估设计属性,方法是直接取平均值,或者从测试产生的值的范围内选择可信的最坏情况下的值。不同的测试。在现场调查程序中收集的信息本质上本质上是多元的,并且在上述早期研究中并未考虑到这方面。本文的目的是为10个常见黏土参数建立无偏转换模型并量化其相关的不确定性。伴随论文\citep{Ching2014686}开发了耦合这些粘土参数的多元概率模型。支持数据库包含来自紧密收集的多个测试的信息。换句话说,每个数据点记录特定位置和深度,即特定采样点的土体信息。请注意,存在测量误差,但是在传统的站点调查程序中无法将测量误差与变换不确定性区分开。因此,本研究中提出的变换不确定性包括一些测量误差,但是对于使用孔隙压力测量(CPTU)进行的圆锥渗透测试,这些误差相对较小\citep{Phoon1999612}。这项研究显然没有考虑土体固有的变异性。原则上,可以通过将多元概率模型(适用于采样“点”)扩展到覆盖整个站点的三维(3D)空间域的向量随机字段来合并固有的土体可变性。因此,从这个意义上说,这项研究的结果是不完整的,但可以认为它们为表征岩土工程变异性铺平了道路,使岩土变异性超越了单变量数据,并与现场调查程序在实际实践中对土体特性的估算更加接近。
    }
    \ParallelPar
    \ParallelLText
    {
        As noted by \citet{Phoon1999612}, the measurement from a geotechnical test is typically not directly applicable to design. Instead, a transformation model is needed to relate the test measurement to an appropriate design parameter. Most transformation models in geotechnical engineering are obtained by empirical or semi-empirical data-fitting using regression analyses. These transformation models are widely adopted in geotechnical engineering practice as a matter of practical expediency. Useful compilations of these models (mostly pairwise correlations) are available in the literature (e.g., \citealp{Kulhawy1990}; \citealp{Mayne2001}). A cursory review of these compilations would reveal a rather bewildering variety and number of models. Most models were developed for a specific geomaterial type and (or) a specific locale.
    }
    \ParallelRText
    {
        正如\citet{Phoon1999612}指出的那样,岩土试验的测量结果通常不适用于设计。 取而代之的是,需要一个转换模型来将测试度量与适当的设计参数相关联。 岩土工程中的大多数转换模型都是通过使用回归分析的经验或半经验数据拟合获得的。 出于实际考虑,这些转换模型已在岩土工程实践中广泛采用。 这些模型的有用汇编(主要是成对相关)可在文献中找到(例如\citealp{Kulhawy1990}; \citealp{Mayne2001})。 粗略地回顾一下这些汇编,将会发现令人困惑的各种模型。 大多数模型是针对特定的土工材料类型和(或)特定的区域设置而开发的。
    }
    \ParallelPar
    \ParallelLText
    {
        It is not judicious to apply these models indiscriminately to other sites without a proper appreciation of geomaterial behaviour and geology. This “site-specific” limitation is a distinctive and fundamental feature of geotechnical engineering practice. Geotechnical design must take cognizance of this limitation to avoid gross oversimplification of “ground truths.” As opposed to site-specific models, \citet{Ching201252} demonstrated the construction of “global” models. Global models are calibrated from global databases covering many sites and geomaterial types. \citet{Ching201252} observed that site-specific models are generally more precise than global models, but they can be significantly biased when applied to another site. On the contrary, global models are less precise than site-specific models, but they are less biased. Their observations are already well appreciated by engineers. The key contribution from \citet{Ching201252} was to demonstrate these observations with statistical rigor using
        a sizeable global database.
    }
    \ParallelRText
    {
        在没有适当了解地球物质行为和地质的情况下,不加选择地将这些模型应用于其他地点并不明智。这种“特定于地点”的限制是岩土工程实践的独特和基本特征。岩土设计必须意识到这一局限性,以避免对“地面真相”的过分简化。与特定地点的模型相反,\citet{Ching201252}演示了“全局”模型的构建。全局模型是根据覆盖许多站点和岩土材料类型的全局数据库进行校准的。 \citet{Ching201252}观察到,特定于站点的模型通常比全局模型更为精确,但将其应用于其他站点时可能会出现明显偏差。相反,全局模型不如特定于站点的模型精确,但它们的偏向性较小。他们的观察已经得到工程师的赞赏。\citet{Ching201252}的主要贡献是通过使用
    }
    \ParallelPar
    \ParallelLText
    {
        Because most transformation models were built based on their own databases, their ranges of application are, in principle, limited to the range of characteristics contained in the databases, e.g., certain soil types, certain range of soil properties (e.g., insensitive clays), and certain geographic locations. It is important to assess their biases and the uncertainties when these models are applied globally (i.e., applying these models outside their range of calibration). In the current paper, a global clay database is compiled and presented. This database consists of data points from 251 studies, covering clay data from 30 regions or countries worldwide. Hence, the range of data covered by this “global” database is broader than that underlying the calibration of existing transformation models in the literature. Ten parameters of clays are of main interest, including three index properties (i.e., Atterberg’s limits); four parameters for effective stresses, shear strength, and sensitivity; and three parameters from piezocone tests (CPTU). This global database is the largest database compiled by the authors thus far in terms of number of data points and number of parameters of interest. Table \ref{table:1} shows the databases compiled by the authors, labeled as (soil type)/(number of parameters of interest)/(number of data points). The current global database is CLAY/10/7490. The first purpose of this paper is to present this large database and verify whether the correlation behaviours in the data points are consistent with those exhibited by existing transformation models in the literature. Most of these models are sitespecific models. The biases and uncertainties in these models will be estimated using the global database. The site-specific models can be applied to a wider range of conditions when their biases are corrected and their transformation uncertainties are  uitably revised.
    }
    \ParallelRText
    {
        由于大多数转化模型都是基于其自身的数据库构建的,因此,其应用范围原则上仅限于数据库中包含的特征范围,例如某些土体类型,某些土体特性范围(例如不敏感的粘土),和某些地理位置。在全球范围内应用这些模型时(即在校准范围之外应用这些模型),评估它们的偏差和不确定性很重要。在当前的论文中,一个全球粘土数据库被编译并呈现出来。该数据库包含来自251个研究的数据点,涵盖了来自全球30个地区或国家的粘土数据。因此,该“全球”数据库所涵盖的数据范围比文献中现有转换模型的标定所依据的范围要广。十个主要参数是黏土,其中包括三个指数属性(即,阿特伯格极限);有效应力,剪切强度和灵敏度的四个参数;以及压电锥测试(CPTU)的三个参数。就数据点的数量和感兴趣的参数的数量而言,该全局数据库是迄今为止作者编写的最大的数据库。表\ref{table:1}显示了作者编译的数据库,标记为(土体类型)/(感兴趣的参数数量)/(数据点数量)。当前的全局数据库是CLAY/10/7490。本文的首要目的是介绍这个大型数据库,并验证数据点中的相关行为是否与文献中现有转换模型所显示的行为一致。这些模型大多数是特定于站点的模型。这些模型中的偏差和不确定性将使用全球数据库进行估算。当特定位置的模型的偏差得到纠正并且适当地修改了转换不确定性时,可以将其应用于更广泛的条件。
    }
    \ParallelPar
>>>>>>> parent of 34d827c... Ching2014
    \begin{table*}[!htb]
    \centering
    \footnotesize
    \caption{Databases compiled by the authors.}
    \addtocounter{table}{-1}
    \vspace{-8pt}
    \renewcommand{\tablename}{表}
    \caption{作者编译的数据库。}
    \vspace{4pt}
    \renewcommand{\tablename}{Table}
    \setlength{\tabcolsep}{1mm}{
    \begin{tabularx}{\textwidth}{XlllllXX}
        \toprule
        \multirow{2}{*}{Database}  & \multirow{2}{*}{Reference}  &  \multirow{2}{*}{Parameters of interest}  &  \multirow{2}{*}{\shortstack{No. of \\data \\points}}    &  \multirow{2}{*}{\shortstack{No. of \\sites or \\studies}}    & \multicolumn{3}{l}{Range of Properties} \\
             &         &     	 &     &     & OCR   & PI      & $S_t$ \\
        \midrule
        CLAY/5/345      & \citet{Ching2012522}  & $\rm{LI},s_u,s_u^{re},\sigma_p',\sigma_v'$        									    & 345   		    &   37 sites    		 & 1$\sim$4   & -     			& \tabincell{l}{Sensitive to \\quick clays} \\
        \specialrule{0em}{2pt}{2pt}
        CLAY/6/535      & \citet{Ching201477}   & \tabincell{l}{$s_u/\sigma_v',\rm{OCR},(q_t-s_v)/\sigma_v',$\\$(q_t-u_2)/\sigma_v',(u_2-u_0)/\sigma_v',B_q$}      				    & 535   		    &   40 sites    		 & 1$\sim$6   & \tabincell{l}{Low to very \\high plasticity}  & \tabincell{l}{Insensitive to \\quick clays} \\
        \specialrule{0em}{2pt}{2pt}
        CLAY/7/6310     & \citet{Ching2013907}  & \tabincell{l}{$s_u~\rm{under~seven~different~}s_u~$\\$\rm{test~types}$}      								    & 6310  		    &   164 studies   		 & 1$\sim$10  & \tabincell{l}{Low to very \\high plasticity}  & \tabincell{l}{Insensitive to \\quick clays} \\
        \specialrule{0em}{2pt}{2pt}
        CLAY/10/7490    & This paper            & \tabincell{l}{$\rm{LL},\rm{PI},\rm{LI},\sigma_v'/P_a,\sigma_p'/P_a,$\\$s_u/\sigma_v',S_t,(q_t-\sigma_v)/\sigma_v',$\\$(q_t-u_2)/\sigma_v',B_q$}      & 7490  		    &   251 studies  		 & 1$\sim$10  & \tabincell{l}{Low to very \\high plasticity}  & \tabincell{l}{Insensitive to \\quick clays} \\
        \bottomrule
    \end{tabularx}}%
    \label{table:1}%
\end{table*}

    \switchcolumn*

    The uncertainties of the resulting global models are inevitably large because they are required to accommodate a broader range of clays, e.g., insensitive, sensitive, and quick clays. It is possible to reduce the transformation uncertainty by considering soil index properties (such as plasticity index and sensitivity) as secondary input (explanatory) parameters. Most of the existing transformation models do not include such secondary parameters. Nonetheless, with the global database compiled in this study, it is possible to augment the existing transformation models with secondary input parameters. The resulting models will still be global, but with less uncertainties. This is the second purpose of this paper. In a companion paper \citep{Ching2014686}, a 10-dimensional multivariate probability distribution coupling these clay parameters is constructed from CLAY/10/7490 and a useful application involving updating the entire bivariate probability distribution of two design parameters from three separate measurements is presented.
    
    \switchcolumn
    
    所产生的全局模型的不确定性不可避免地很大,因为需要它们适应更大范围的粘土,例如不敏感,敏感和快速的粘土。通过将土体指数属性(例如可塑性指数和敏感性)视为辅助输入(解释性)参数,可以减少变换不确定性。现有的大多数转换模型都不包含此类辅助参数。但是,通过本研究中编译的全局数据库,可以使用辅助输入参数来扩展现有的转换模型。所得模型仍将是全局模型,但不确定性较小。这是本文的第二个目的。在随附的论文中\citep{Ching2014686},从CLAY/10/7490构建了耦合这些黏土参数的10维多元概率分布,并呈现了一个有用的从三个独立的测量值中更新两个设计参数的整体二元概率分布的应用。

\end{paracol}